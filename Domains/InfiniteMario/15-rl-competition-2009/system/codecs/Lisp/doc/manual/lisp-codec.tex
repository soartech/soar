\documentclass[11pt,a4paper,dvipdfm]{article}

\usepackage[footnotesep=8mm]{geometry}
\usepackage{verbatim}
\usepackage{fullpage}
\usepackage[colorlinks=true,
            linkcolor=cyan,
            urlcolor=blue]{hyperref}

% do not indent paragraphs
\setlength{\parindent}{0pt}
% set space between paragraphs
\setlength{\parskip}{10pt}

\newcommand{\selfref}[1]{\href{#1}{#1}}
\newcommand{\footref}[2]{\textsl{#1}\footnote{\selfref{#2}}}
\newcommand{\prompttext}[1]{\texttt{#1}}
\newcommand{\shprompt}[1]{\prompttext{\$ #1}}
\newcommand{\lispprompt}[1]{\prompttext{* #1}}

\begin{document}

\title{RL-Glue Lisp Codec 1.0 Manual}
\author{Gabor Balazs \\ gabalz@rl-community.org}
\date{}

\maketitle
\setcounter{tocdepth}{2}
\tableofcontents
\newpage

%%%%%%%%%%%%%%%%%%%%%%%%%%%%%%%%%%%%%%%%%%%%%%%%%%%%%%%%%%%%%%%%%%%%%%%%%%%%%%%

\section{Introduction}

RL-Glue codecs provide TCP/IP connectivity to the RL-Glue reinforcement
learning software library. This codec makes possible to create agent,
environment and experiment programs in the Common Lisp programming language.

For general information and motivation about the
\footref{RL-Glue library}{http://glue.rl-community.org/}, please refer to the
documentation provided with that project.

This software is licensed under the
\footref{Apache 2.0 license}{http://www.apache.org/licenses/LICENSE-2.0.html}.
We are not lawyers, but our intention is that this codec should be used however
it is useful. We would appreciate to hear what you are using it for, and to get
credit if appropriate.

\subsection{Software requirements}
\hypertarget{softreqs}{}

Required libraries for the codec.
\begin{itemize}
\item[] split-sequence (\selfref{http://www.cliki.net/SPLIT-SEQUENCE})
\item[] usocket (\selfref{http://common-lisp.net/project/usocket/})
\end{itemize}

\subsection{Supported Lisp implementations}

\begin{itemize}
\item[] SBCL (\selfref{http://www.sbcl.org/})
\item[] CMUCL (\selfref{http://www.cons.org/cmucl/})
\item[] Lispworks (\selfref{http://www.lispworks.com/})
\item[] Allegro CL (\selfref{http://www.franz.com/products/allegrocl/})
\item[] CLISP (\selfref{http://www.gnu.org/software/clisp/})
\item[] Scieneer (\selfref{http://www.scieneer.com/scl/})
\item[] CCL / OpenMCL (\selfref{http://www.clozure.com/clozurecl.html})
\end{itemize}

\subsection{Getting the codec}

The codec can be downloaded either as a tarball or can be checked out of the
subversion repository.

The tarball distributions can be found here: \\
\selfref{http://code.google.com/p/rl-glue-ext/wiki/Lisp}

The subversion trunk can be checked out this way: \\
\shprompt{svn co
          http://rl-glue-ext.googlecode.com/svn/trunk/projects/codecs/Lisp
          lisp-codec}

\subsection{Installation}

The codec provides ASDF packages, so the
\footref{ASDF library}{http://common-lisp.net/project/asdf/} has to be set up
for your Lisp implementation. If you want to know more about it, take a look
at our \hyperlink{asdfintro}{quick ASDF introduction.}

The main ASDF packages are \prompttext{rl-glue-codec} and
\prompttext{rl-glue-utils}. The former provides the connectivity with the
RL-Glue component, the latter contains utilities which can be helpful during
using the codec. The \prompttext{rl-glue-examples} package contains the
agent, environment and experiment examples described later in this document.

The codec is distributed in three different ways described in the following
sections. Choose according to your needs and knowledge, but if you remain
unsure at the end, we recommend the full distribution.

\subsubsection{Full distribution}

\textsl{Recommended for users without ASDF experience.}

This distribution contains a pre-configured ASDF environment with all the
libraries required to use the codec. It also provides the manual and the
API reference documentation. You don't need any ASDF installation and
configuration knowledge to use this distribution.

The installation is very simple.

Choose the install destination. \\
\shprompt{cd /path/to}

Unpack the files. \\
\shprompt{tar -zxf lisp-codec-full-<version>.tar.gz} \\

Then you are able to use the codec.

Start your Lisp. \\
\shprompt{lisp} \\
\lispprompt{}

Initialize the ASDF sustem. \\
\lispprompt{(load \#p"/path/to/lisp-codec-full/setup")} \\

Now you are ready to load the rl-glue ASDF packages. The ASDF system will
compile them automatically on the first time to reduce the loading time of
the later occasions. Continue by reading the
\hyperlink{usage}{usage instructions}.

\subsubsection{ASDF distribution}

\textsl{Recommended for users with ASDF experience.}

If you are an experienced ASDF user, you probably have an installed and
configured ASDF system. Then this distribution is for you, it contains
the \prompttext{rl-glue-codec}, \prompttext{rl-glue-utils} and
\prompttext{rl-glue-examples} ASDF packages which can be installed as any
other ASDF package. It also provides the manual and the API reference
documentation.

If you choose this distribution, you have to install the
\hyperlink{softreqs}{library dependencies} on your own. Of course,
you can use a sophisticated installer (like
\footref{ASDF-Install}{http://www.cliki.net/ASDF-Install}
or
\footref{clbuild}{http://common-lisp.net/project/clbuild/})
which can handle this problem.

\subsubsection{Developer distribution}

\textsl{Recommended for Lisp codec developers.}

This distribution is a snapshot of the Lisp codec development directory.
It contains all the source code, tests, documentation and development tools.
If you choose this, you have to do the installation as in case of the ASDF
distribution.

\subsection{Uninstallation}

Just delete what you don't need anymore.

If you used a sophisticated tool for the installation (like
ASDF-Install or clbuild), you should use its uninstall method.

If you have used ASDF-Binary-Locations, don't forget to remove the
\prompttext{fasl} files.

\subsection{Credits and acknowledgment}

G\'abor Bal\'azs wrote the Lisp codec, that's me!

I would like thank the lispers on the
\footref{\prompttext{\#lisp} IRC channel}{http://www.cliki.net/IRC}
on \textsl{irc.freenode.net} of their great help in many Lisp specific
questions.

\subsubsection{Contributing}

If you would like to become a member of this project and contribute 
updates/changes to the code, please send a message to rl-glue@googlegroups.com.

%%%%%%%%%%%%%%%%%%%%%%%%%%%%%%%%%%%%%%%%%%%%%%%%%%%%%%%%%%%%%%%%%%%%%%%%%%%%%%%

\section{Using the codec}
\hypertarget{usage}{}

This section describes how the codec can be used to create agents, environments
and experiments in Lisp, and how they can be glued into a running system.

\subsection{Packages}

The codec has an own (ASDF) package named \prompttext{rl-glue-codec}. \\
\lispprompt{(asdf:oos 'asdf:load-op :rl-glue-codec)}

You can use the proper package qualified symbol names, e.g. \\
\lispprompt{(defclass my-agent (rl-glue-codec:agent) \ldots)} \\
\lispprompt{(defmethod rl-glue-codec:env-init ((env my-env)) \ldots)}

Or you can import all the symbols into your package by the \prompttext{:use}
directive, and use the codec symbols without any package qualification
(the further examples will assume this case). \\
\lispprompt{(defpackage :my-package (:use :rl-glue \ldots)~\ldots)}

There are a few utilities for the codec which can be useful. These can be
accessed in the \prompttext{rl-glue-utils} (ASDF) package. More information
about them can be found in \hyperlink{rlutils}{their section}.

\subsection{Types}

There is an abstract data type, \prompttext{rl-abstract-type}, which can
contain integers, floating point numbers and a character string. Its slots can
be accessed by the \prompttext{int-array}, \prompttext{float-array} and
\prompttext{char-string} functions. There are two macros for number array
creation, \prompttext{make-int-array} and \prompttext{make-float-array},
which automatically set the type of the contained elements according to the
codec requirements. The usage of them is strongly suggested.

\begin{center}
\begin{tabular}{ll}
    observation         & rl-abstract-type \\
    action              & rl-abstract-type \\
    reward              & double-float \\
    terminal            & boolean \\
    task specification  & string \\
    state key           & rl-abstract-type \\
    random seed key     & rl-abstract-type \\
\end{tabular}
\end{center}

\subsection{Agents}

On writing an agent first, you have to create an own agent class, e.g. \\
\lispprompt{(defclass my-agent (agent) \ldots)}

Second, implement the following methods for it. \\
\lispprompt{(defmethod agent-init ((agent my-agent) task-spec) \ldots)} \\
\lispprompt{(defmethod agent-start ((agent my-agent) first-observation) \ldots)} \\
\lispprompt{(defmethod agent-step ((agent my-agent) reward observation) \ldots)} \\
\lispprompt{(defmethod agent-end ((agent my-agent) reward) \ldots)} \\
\lispprompt{(defmethod agent-cleanup ((agent my-agent)) \ldots)} \\
\lispprompt{(defmethod agent-message ((agent my-agent) input-message) \ldots)}

A detailed description of the methods can be obtained this way. \\
\lispprompt{(documentation \#'<method-name> 'function)}

When your agent is ready, you can run it. \\
\lispprompt{(run-agent (make-instance 'my-agent) \\
\mbox{~~~~~~~~~~~~~}:host "192.168.1.1" \\
\mbox{~~~~~~~~~~~~~}:port 4096 \\
\mbox{~~~~~~~~~~~~~}:retry-timeout 10)}

It will try to connect your agent to an RL-Glue component listening on
\prompttext{192.168.1.1} and port \prompttext{4096}, waiting 10 seconds
between the trials. Its detailed description can be checked this way. \\
\lispprompt{(documentation \#'run-agent 'function)}

It will prompt something like this, where each dot denotes a connection
trial. \\
\prompttext{RL-Glue Lisp Agent Codec Version 1.0, Build 414} \\
\prompttext{\mbox{~~~~~~~~}Connecting to 192.168.1.1:4096 ....~ok}

\subsection{Environments}

On writing an environment, first you have to create an own environment class,
e.g. \\
\lispprompt{(defclass my-env (environment) \ldots)}

Second, implement the following methods for it. \\
\lispprompt{(defmethod env-init ((env my-env)) \ldots)} \\
\lispprompt{(defmethod env-start ((env my-env)) \ldots)} \\
\lispprompt{(defmethod env-step ((env my-env) action) \ldots)} \\
\lispprompt{(defmethod env-cleanup ((env my-env)) \ldots)} \\
\lispprompt{(defmethod env-message ((env my-env) input-message) \ldots)}

A detailed description of the methods can be obtained this way. \\
\lispprompt{(documentation \#'<method-name> 'function)}

When your environment is ready, you can run it. \\
\lispprompt{(run-env (make-instance 'my-env) \\
\mbox{~~~~~~~~~~~}:host "192.168.1.1" \\
\mbox{~~~~~~~~~~~}:port 4096 \\
\mbox{~~~~~~~~~~~}:retry-timeout 10)}

It will try to connect your environment to an RL-Glue component listening on
\prompttext{192.168.1.1} and port \prompttext{4096}, waiting 10 seconds
between the trials. Its detailed description is here. \\
\lispprompt{(documentation \#'run-env 'function)}

It will prompt something like this, where each dot denotes a connection
trial. \\
\prompttext{RL-Glue Lisp Environment Codec Version 1.0, Build 414} \\
\prompttext{\mbox{~~~~~~~~}Connecting to 192.168.1.1:4096 ....~ok}

\subsection{Experiments}

On writing an experiment, first you have to create an own experiment class,
e.g. \\
\lispprompt{(defclass my-exp (experiment) \ldots)}

Second, implement your experiment. For this the codec provides client
functions which hides the necessary buffer handling and network operation.
These are the following.

\prompttext{rl-init},
\prompttext{rl-start},
\prompttext{rl-step},
\prompttext{rl-cleanup},
\prompttext{rl-close},
\prompttext{rl-return},
\prompttext{rl-num-steps}, \\
\prompttext{rl-num-episodes},
\prompttext{rl-episode},
\prompttext{rl-agent-message},
\prompttext{rl-env-message}.

\ \\
A detailed description of these functions can be obtained this way. \\
\lispprompt{(documentation \#'<function name> 'function)}

Do not forget to call the \prompttext{rl-close} function at the end of your
experiment, because it closes the network connection and so terminates the
RL-Glue session.

\subsection{Running}

After you have an agent, an environment and an experiment, which could be
written on any of the supported languages, you can connect them by RL-Glue.

First start the server. \\
\shprompt{rl\_glue}

You should see this kind of output on the server side. \\
\prompttext{RL-Glue Version 3.0, Build 909 \\
RL-Glue is listening for connections on port=4096 \\
\mbox{~~~~~~~~}RL-Glue ::~Agent connected. \\
\mbox{~~~~~~~~}RL-Glue ::~Environment connected. \\
\mbox{~~~~~~~~}RL-Glue ::~Experiment connected.}

Then start the agent, the environment and the experiment.

The output of the Lisp agent. \\
\prompttext{RL-Glue Lisp Agent Codec Version 1.0, Build 414 \\
\mbox{~~~~~~~~}Connecting to 127.0.0.1:4096 .~ok}

The output of the Lisp environment. \\
\prompttext{RL-Glue Lisp Environment Codec Version 1.0, Build 414 \\
\mbox{~~~~~~~~}Connecting to 127.0.0.1:4096 .~ok}

The output of the Lisp experiment. \\
\prompttext{RL-Glue Lisp Experiment Codec Version 1.0, Build 414 \\
\mbox{~~~~~~~~}Connecting to 127.0.0.1:4096 .~ok}

\subsection{Utilities}
\hypertarget{rlutils}{}

The utilities has an own (ASDF) package named \prompttext{rl-glue-utils}. \\
\lispprompt{(asdf:oos 'asdf:load-op :rl-glue-utils)}

\subsubsection{Task specification parser}

This is a parser for the
\footref{task specification language}
{http://glue.rl-community.org/Home/rl-glue/task-spec-language}. Only the
RL-Glue version 3.0 specification type is supported.

The parser can return a \prompttext{task-spec} object from a specification
string, e.g. \\
\lispprompt{(parse-task-spec \\
\mbox{~~~~}"VERSION RL-Glue-3.0 PROBLEMTYPE episodic DISCOUNTFACTOR 1 \\
\mbox{~~~~~}OBSERVATIONS INTS (3 0 1) ACTIONS DOUBLES (3.2 6.5) CHARCOUNT 50 \\
\mbox{~~~~~}REWARDS (-1.0 1.0) EXTRA extra specification") \\
\#<TASK-SPEC>}

The \prompttext{task-spec} object has the following slots.

\prompttext{version}, \prompttext{problem-type}, \prompttext{discount-factor}, \\
\prompttext{int-observations}, \prompttext{float-observations},
\prompttext{char-observations}, \\ \prompttext{int-actions},
\prompttext{float-actions}, \prompttext{char-actions}, \\
\prompttext{rewards}, \prompttext{extra-spec}

Their names are appropriately show their functionalities according to the task
specification documentation. The \prompttext{int-} and \prompttext{float-}
observation and action slot values contain \prompttext{int-range} and
\prompttext{float-range} objects appropriately. These have the
\prompttext{repeat-count}, \prompttext{min-value} and \prompttext{max-value}
slots. For the latter two there are three special symbols which are
\prompttext{'-inf}, \prompttext{'+inf} and \prompttext{'unspec}. They are used
to represent the NEGINF, POSINF and UNSPEC specification keywords.

The \prompttext{task-spec} type supports the \prompttext{to-string} generic
function with which it can be converted to a string. There are also two other
helper functions, namely \prompttext{across-ranges} and
\prompttext{ranges-dimension}, which can be useful for range vector handling.
Their documentation can be checked by the \prompttext{documentation} function.

\lispprompt{(documentation 'rl-glue-utils:across-ranges 'function)} \\
\lispprompt{(documentation 'rl-glue-utils:ranges-dimension 'function)}

\subsection{Examples}
\hypertarget{examples}{}

There is an (ASDF) package named \prompttext{rl-glue-examples} located
in the \prompttext{example} directory. You have to load it before you
run the examples. \\
\lispprompt{(pushnew \#p"/path/to/lisp-codec/examples/" asdf:*central-registry*)} \\
\lispprompt{(asdf:oos 'asdf:load-op :rl-glue-examples)}

\subsubsection{The skeleton example}

The \prompttext{skeleton} example has a very simple agent, environment
and experiment of which duty is only an introduction about the usage of
the codec for the user. The environment is a ``discrete line'' with 21
states and the agent starts from the middle and can go left and right.
It randomly selects the actions and rewarded differently on the ends of
the line. The experiment tries to cover the presentation of the used
features.

To run the example you have to do the following. \\
Running the RL-Glue core. \\
\shprompt{rl\_glue} \\
Starting the agent (in the first Lisp thread). \\
\lispprompt{(rl-glue-skeleton:start-agent)} \\
Starting the environment (in the second Lisp thread). \\
\lispprompt{(rl-glue-skeleton:start-environment)} \\
Starting the experiment (in the third Lisp thread). \\
\lispprompt{(rl-glue-skeleton:start-experiment)}

\subsubsection{The mines-sarsa example}

The \prompttext{mines-sarsa} example has an episodic environment. At the
beginning of each episode the agents is put somewhere onto the mine field.
Its goal is to find the exit point without stepping to a mine. The rewards
are $-1$ for each intermediate step, $-100$ for stepping to a mine and $+10$
for reaching the exit. The example contains a simple sarsa agent with epsilon
greedy action selection policy and a simple lookup table to store the action
values. A few basic functionalities like turning learning or exploration
on/off, saving the value function and setting the starting position are
implemented by the message system of RL-Glue.

To run the example you have to do the following. \\
Running the RL-Glue core. \\
\shprompt{rl\_glue} \\
Starting the agent (in the first Lisp thread). \\
\lispprompt{(rl-glue-mines-sarsa:start-agent)} \\
Starting the environment (in the second Lisp thread). \\
\lispprompt{(rl-glue-mines-sarsa:start-environment)} \\
Starting the experiment (in the third Lisp thread). \\
\lispprompt{(rl-glue-mines-sarsa:start-experiment)}

\section{Lisp related topics}

We provide here example solutions for a few Lisp related problems.

\subsection{Quick introduction to ASDF}
\hypertarget{asdfintro}{}

ASDF (Another System Definition Facility) is a library with which another
Lisp libraries can be distributed. The home page of ASDF is here
\selfref{http://common-lisp.net/project/asdf/}, and many ASDF libraries can
be found here \selfref{http://www.cliki.net/Library}.

Sometimes ASDF is included by the Lisp implementation (e.g. in case of SBCL
or CCL), in other cases you can download it from its home page. It is only a
single Lisp file which does not depend on anything, so it can be simply loaded.

The ASDF facility usually has two directories. The source directory in which
the packages are stored, and the system directory from which symlinks are
pointed to the .asd system definition files. In this manual we refer to the
former by \prompttext{asdf-source-dir} and to the latter by
\prompttext{asdf-system-dir}. If your file system does not support symbolic
links (e.g. in case of 
\footref{FAT}{http://en.wikipedia.org/wiki/File\_Allocation\_Table} or
\footref{NTFS}{http://en.wikipedia.org/wiki/NTFS}) or you just don't
want to use symbolic links, take a look at the
\hyperlink{nosymlink}{Using ASDF without symbolic links} section. Be aware
of that \footref{Cygwin}{http://www.cygwin.com/} supports symbolik links,
but it probably works only with Lisps compiled under it, and might not
function well with the pre-compiled Windows Lisp binaries.

To set up ASDF create these directories, e.g.: \\
\shprompt{mkdir /.../asdf} \\
\shprompt{mkdir /.../asdf/source} \\
\shprompt{mkdir /.../asdf/system}

Put the ASDF Lisp file, \prompttext{asdf.lisp} into \prompttext{/.../asdf} and
create or extend your Lisp init file (check the manual of your Lisp
about it) by the followings to prevent typing it all the time. \\
\prompttext{(load \#p"/.../asdf/asdf.lisp")} \\
\prompttext{(pushnew \#p"/.../asdf/system/" asdf:*central-registry*)} \\
The ending slash is important in the path, don't forget it!

Then you are ready to install ASDF libraries by downloading and unpacking them
into the \prompttext{asdf-source-dir} directory and creating a symlink to the
appropriate (there can be more, e.g. for testing) .asd file(s) from the
\prompttext{asdf-system-dir} directory.

You can compile and load the installed libraries by this command. \\
\lispprompt{(asdf:oos 'asdf:load-op :library-name)}

By default the compiled Lisp file is placed next to the source which is
usually unwanted because of two reasons. First it messes up the source
directory, but more importantly if the source directory was created by an
other user (e.g. in case of system-wide installations), the current user may
not have write access for it and on such an attempt an error is signalled.

The cure for this problem is the ASDF-Binary-Locations library. Its home
page is located here
\selfref{http://common-lisp.net/project/asdf-binary-locations/}. It stores
the compiled files under a specifiable directory in the same structure as
the sources are organised. \\

You can install this library using the above described method. Download and
unpack it to \prompttext{asdf-source-dir}, then create the appropriate
symbolic link.

\shprompt{cd asdf-system-dir} \\
\shprompt{ln -s
/asdf-source-dir/asdf-binary-locations/asdf-binary-locations.asd \\
\mbox{~~~~~~~~}asdf-binary-location.asd}

After the installation, you might want to create a directory for the
compilation result files, e.g. \\
\shprompt{mkdir /.../fasls}

And add the followings to your Lisp init file. \\
\prompttext{(setf asdf:*centralize-lisp-binaries* t)} \\
\prompttext{(setf asdf:*default-toplevel-directory* \#p"/.../fasls/")}

\hypertarget{nosymlink}{\subsection{Using ASDF without symbolic links}}

This manual always considers that you are using symlinks, but if you apply
this method, just ignore those parts, everything should work without them.

When symbolic links are not available or not welcomed to use, its system
definition search functionality can be redefined to find the \prompttext{.asd}
files recursively under the \prompttext{asdf:*central-registry*} directories.
It will let you simply ``drop-in'' the new packages into these directories,
and they will be available for loading without any further steps. For this
you might want to add the following to the Lisp init file just after loading
the \prompttext{asdf.lisp} file.

\prompttext{(in-package :asdf) \\
(defvar *subdir-search-registry* '(\#p"/my/lisp/libraries/") \\
\mbox{~~}"List of directories to search subdirectories within.") \\
(defvar *subdir-search-wildcard* :wild \\
\mbox{~~}"Value of :wild means search only one level of subdirectories; \\
\mbox{~~~}value of :wild-inferiors means search all levels of subdirectories \\
\mbox{~~~}(I don't advise using this in big directories!)") \\
(defun sysdef-subdir-search (system) \\
\mbox{~~}(let ((latter-path (make-pathname :name (coerce-name system) \\
\mbox{~~~~~~~~~~~~~~~~~~~~~~~~~~~~~~~~~~~~}:directory (list :relative \\
\mbox{~~~~~~~~~~~~~~~~~~~~~~~~~~~~~~~~~~~~~~~~~~~~~~~~~~~~~}*subdir-search-wildcard*) \\
\mbox{~~~~~~~~~~~~~~~~~~~~~~~~~~~~~~~~~~~~}:type "asd" \\
\mbox{~~~~~~~~~~~~~~~~~~~~~~~~~~~~~~~~~~~~}:version :newest \\
\mbox{~~~~~~~~~~~~~~~~~~~~~~~~~~~~~~~~~~~~}:case :local))) \\
\mbox{~~~~}(dolist (d *subdir-search-registry*) \\
\mbox{~~~~~~}(let* ((wild-path (merge-pathnames latter-path d)) \\
\mbox{~~~~~~~~~~~~~}(files (directory wild-path))) \\
\mbox{~~~~~~~~~}(when files \\
\mbox{~~~~~~~~~~~}(return (first files))))))) \\
(pushnew 'sysdef-subdir-search *system-definition-search-functions*) \\
(in-package :cl-user)}

This method is published on the \selfref{http://www.cliki.net/asdf} page
in the \textsl{Alternative Sysdef Search functionality} section.

\end{document}

